\documentclass[10pt,A4]{article}

%============================================================================%
%
%	DOCUMENT DEFINITION
%
%============================================================================%
%----------------------------------------------------------------------------------------
%	ENCODING
%----------------------------------------------------------------------------------------
\usepackage[utf8]{inputenc}		
%----------------------------------------------------------------------------------------
%	LOGIC
%----------------------------------------------------------------------------------------
\usepackage{xstring, xifthen}
%----------------------------------------------------------------------------------------
%	FONT BASICS
%----------------------------------------------------------------------------------------
\usepackage{mathptmx}
% set font default
\renewcommand*\familydefault{\sfdefault} 	
\usepackage[T1]{fontenc}
% more font size definitions
\usepackage{moresize}
%----------------------------------------------------------------------------------------
% use to vertically center icon
% credits to: http://tex.stackexchange.com/questions/7219/how-to-vertically-center-two-images-next-to-each-other
\newcommand{\vcenteredinclude}[1]{\begingroup
\setbox0=\hbox{\includegraphics[width=4mm, height=5mm]{#1}}%
\parbox{\wd0}{\box0}\endgroup}
%----------------------------------------------------------------------------------------
%	PAGE LAYOUT  DEFINITIONS
%----------------------------------------------------------------------------------------
% we use paracol to display breakable two columns
\usepackage{paracol}
% define page styles using geometry
\usepackage[a4paper]{geometry}
% remove all possible margins
\geometry{top=1cm, bottom=1cm, left=1cm, right=1cm}
\usepackage{fancyhdr}
\pagestyle{empty}
% indentation is zero
\setlength{\parindent}{0mm}
%----------------------------------------------------------------------------------------
%	TABLE /ARRAY DEFINITIONS
%---------------------------------------------------------------------------------------- 
% extended aligning of tabular cells
\usepackage{array}
%----------------------------------------------------------------------------------------
%	GRAPHICS DEFINITIONS
%---------------------------------------------------------------------------------------- 
\usepackage{graphicx}
\usepackage{tikz}				
\usetikzlibrary{shapes, backgrounds, mindmap, trees}
%----------------------------------------------------------------------------------------
%	Color DEFINITIONS
%---------------------------------------------------------------------------------------- 
\usepackage{transparent}
\usepackage{color}
% primary color
\definecolor{maincol}{RGB}{ 225, 0, 0 }
% dark color
\definecolor{darkcol}{RGB}{ 70, 70, 70 }
% light color
\definecolor{lightcol}{RGB}{245,245,245}

% Package for links, must be the last package used
\usepackage[hidelinks]{hyperref}
% returns minipage width minus two times \fboxsep
% to keep padding included in width calculations
% can also be used for other boxes / environments
\newcommand{\mpwidth}{\linewidth-\fboxsep-\fboxsep}
%============================================================================%
%
%	CV COMMANDS
%
%============================================================================%
%----------------------------------------------------------------------------------------
%	 CV LIST
%----------------------------------------------------------------------------------------
% renders a standard latex list but abstracts away the environment definition (begin/end)
\newcommand{\cvlist}[1] {
	\begin{itemize}{#1}\end{itemize}
}
%----------------------------------------------------------------------------------------
%	 CV TEXT
%----------------------------------------------------------------------------------------
% base class to wrap any text based stuff here. Renders like a paragraph.
% Allows complex commands to be passed, too.
% param 1: *any
\newcommand{\cvtext}[1] {
	\begin{tabular*}{1\mpwidth}{p{0.98\mpwidth}}
		\parbox{1\mpwidth}{#1}
	\end{tabular*}
}
%----------------------------------------------------------------------------------------
%	CV SECTION AND SUBSECTION
%----------------------------------------------------------------------------------------
% Renders a a CV section headline with a nice underline in main color.
% param 1: section title
\newcommand{\cvsection}[1] {
	\vspace{8pt}
	\cvtext{
		\textbf{\LARGE{\textcolor{darkcol}{\texttt{#1}}}}\\[-7pt]
		\textcolor{maincol}{ \rule{0.075\textwidth}{1.25pt} }
	}
}

\newcommand{\cvsubsection}[1] {
	\vspace{10pt}
	\cvtext{
		\textit{\large{\textcolor{darkcol}{\textbf{#1}}}}
	}
	\vspace{5pt}
}
%----------------------------------------------------------------------------------------
%	META SKILL
%----------------------------------------------------------------------------------------
% Renders a progress-bar to indicate a certain skill in percent.
% param 1: name of the skill / tech / etc.
% param 2: level (for example in years)
% param 3: percent, values range from 0 to 1
\newcommand{\cvskill}[3] {
  \vspace{2pt}
  \hspace{2.5pt}
	\begin{tabular*}{1\mpwidth}{p{0.3\mpwidth}  r}
 		\textcolor{black}{\texttt{#1}} & \begin{tikzpicture}[scale=1,rounded corners=2pt,very thin]
		  \fill [maincol] (0,0) rectangle (#3\mpwidth, 0.1);
  	\end{tikzpicture}%
	\end{tabular*}%
	\vspace{-4pt}
}

\newcommand{\cvskilltool}[3] {
  \vspace{2pt}
  \hspace{2.5pt}
  \begin{tabular*}{1\mpwidth}{p{0.25\mpwidth} p{0.37\mpwidth}  r}
  \textcolor{black}{\texttt{#1}} & {\tiny \textit{\textcolor{maincol}{#2}}} &	\begin{tikzpicture}[scale=1,rounded corners=2pt,very thin]
		  \fill [maincol] (0,0) rectangle (#3\mpwidth, 0.1);
  	\end{tikzpicture}%
	\end{tabular*}%
	\vspace{-4pt}
}

\newcommand{\cvcontact}[3] {
  \vspace{4pt}
  \hspace{1pt}
	\begin{tabular*}{1\mpwidth}{p{0.05\mpwidth}  r}
 		\vcenteredinclude {#2} & \textcolor{black}{#1}
	\end{tabular*}%
	\vspace{2pt}
}

\newcommand{\cvsociallink}[3] {
  \vspace{10pt}
  \hspace{1.5pt}
  \begin{tabular*}{1\mpwidth}{p{0.36\mpwidth} p{0.36\mpwidth}  r}
  \href{https://github.com/JBthePenguin/}{\includegraphics[width=4mm, height=4.5mm]{#1}} & \href{https://twitter.com/JBthePenguin}{\includegraphics[width=4mm, height=4.5mm]{#2}} &	\href{https://www.linkedin.com/in/jean-baptiste-labaty-a5b084155/}{\includegraphics[width=4mm, height=4.5mm]{#3}}
	\end{tabular*}%
	\vspace{-40pt}
}
%----------------------------------------------------------------------------------------
%	 CV EVENT
%----------------------------------------------------------------------------------------
% Renders a table and a paragraph (cvtext) wrapped in a parbox (to ensure minimum content
% is glued together when a pagebreak appears).
% Additional Information can be passed in text or list form (or other environments).
% the work you did
% param 1: time-frame i.e. Sep 14 - Jan 15 etc.
% param 2:	 event name (job position etc.)
% param 3: Customer, Employer, Industry
% param 4: Short description
% param 5: work done (optional)
\newcommand{\cvrealase}[5] {
	% we wrap this part in a parbox, so title and description are not separated on a pagebreak
	% if you need more control on page breaks, remove the parbox
	\vspace{4pt}
	\parbox{\mpwidth}{
		\begin{tabular*}{1\mpwidth}{p{0.7\mpwidth} r}
	 		\textcolor{black}{\textbf{#2}} & \colorbox{maincol}{\makebox[0.25\mpwidth]{\textcolor{white}{#1}}} \\
			\textcolor{maincol}{\textbf{#3}} & \\
		\end{tabular*}
		\ifthenelse{\isempty{#4}}{}{
			\cvtext{#4}\\
		}
	  \vspace{4pt}
	}
  \vspace{-17.35pt}  
	\ifthenelse{\isempty{#5}}{}{
		\vspace{-1.5pt}
		{#5}
	}
	\vspace{14pt}
}

\newcommand{\cvnewevent}[4] {
	% we wrap this part in a parbox, so title and description are not separated on a pagebreak
	% if you need more control on page breaks, remove the parbox
	\vspace{4pt}
	\parbox{\mpwidth}{
		\begin{tabular*}{1\mpwidth}{p{0.625\mpwidth}  r}
	 		\textcolor{black}{\textbf{#2}} & \colorbox{maincol}{\makebox[0.325\mpwidth]{\textcolor{white}{{\small #1}}}} \\[0pt]
			\textcolor{maincol}{\textbf{#3}} & \\[-2.5pt]
		\end{tabular*}
		\cvtext{#4}
	}
	\vspace{-5pt}
}

\newcommand{\cvneweventdip}[3] {
  % we wrap this part in a parbox, so title and description are not separated on a pagebreak
	% if you need more control on page breaks, remove the parbox
	\vspace{4pt}
	\parbox{\mpwidth}{
		\begin{tabular*}{1\mpwidth}{p{0.625\mpwidth}  r}
	 		\textcolor{black}{\textbf{#2}} & \colorbox{maincol}{\makebox[0.325\mpwidth]{\textcolor{white}{{\small #1}}}} \\[0pt]
			\textcolor{maincol}{\textbf{#3}} & \\[-2.5pt]
		\end{tabular*}
	}
	\vspace{-5pt}
}
%----------------------------------------------------------------------------------------
%	 CV META EVENT
%----------------------------------------------------------------------------------------
% Renders a CV event on the sidebar
% param 1: title
% param 2: subtitle (optional)
% param 3: customer, employer, etc,. (optional)
% param 4: info text (optional)
\newcommand{\cvmetaevent}[4] {
	\textcolor{maincol} {\cvtext{\textbf{\begin{flushleft}#1\end{flushleft}}}}
	\ifthenelse{\isempty{#2}}{}{
	\textcolor{darkcol} {\cvtext{\textbf{#2}} }
	}
	\ifthenelse{\isempty{#3}}{}{
		\cvtext{{ \textcolor{darkcol} {#3} }}\\
	}
	\cvtext{#4}\\[14pt]
}
%============================================================================%
%
%
%
%	DOCUMENT CONTENT
%
%
%
%============================================================================%
\begin{document}
\columnratio{0.31}
\setlength{\columnsep}{2.2em}
\setlength{\columnseprule}{1.5pt}
\colseprulecolor{darkcol}
\begin{paracol}{2}
%============================================================================%
%
%	Left side
%
%============================================================================%
\begin{leftcolumn}
% HEADER
\fcolorbox{white}{darkcol}{\begin{minipage}[c][1.5cm][c]{1\mpwidth}
\vspace{10pt}
	\begin {center}
	  \Large{ \textcolor{white} {Jean-Baptiste Labaty} } \\[-10pt]
		\textcolor{white}{ \rule{0.2\textwidth}{1pt} } \\
		\large{ \textbf{ \textcolor{white}{ Développeur d'application } } }
	\end {center}
\end{minipage}} \\[14pt]
\vspace{-10pt}
%---------------------------------------------------------------------------------------
%	SKILLS
%----------------------------------------------------------------------------------------
\cvsection{Compétences}

\cvsubsection{Backend}

\cvskill{Python} {} {0.6}

\cvskill{Bash} {} {0.3}

\cvsubsection{Frontend}

\cvskill{Html/CSS} {} {0.45}

\cvskill{Javascript} {} {0.45}

\cvsubsection{Frameworks}

\cvskill{Django} {} {0.6}

\cvskill{Flask} {} {0.3}

\cvskill{Bootstrap} {} {0.45}

\cvskill{JQuery} {} {0.45}

\cvsubsection{Base de données}

\cvskill{PostgreSQL} {} {0.6}

\cvskill{MariaDB} {} {0.3}

\cvskill{SQLite} {} {0.3}

\cvsubsection{Système d'exploitation}

\cvskill{Linux} {} {0.45}

\cvsubsection{Outils}

\cvskilltool{Git} {Versionning} {0.225}

\cvskilltool{TravisCI} {Intégration continue} {0.15} 

\cvskilltool{Sentry} {Monitoring} {0.15}

\cvskilltool{NewRelic} {Monitoring} {0.15}

\cvskilltool{UML} {Conception} {0.15}

\cvskilltool{LaTeX} {Documentation} {0.15}

\vspace{10pt}
%---------------------------------------------------------------------------------------
%	CONTACT
%----------------------------------------------------------------------------------------
\cvsection{Contact}

\vspace{2.5pt}

\cvcontact{Toulouse} {img/mapmarker.png} {}

\cvcontact{06 45 62 61 20} {img/phone.png} {}

\cvcontact{\href{mailto:jbthepenguin@netcourrier.com}{\textit{\underline{jbthepenguin@netcourrier.com}}}} {img/email.png} {}

\cvcontact{\href{https://jbthepenguin.github.io/}{\textit{\underline{jbthepenguin.github.io}}}} {img/site.png} {}

\cvsociallink{img/github.png} {img/twitter.png} {img/linkedin.png}

\end{leftcolumn}

%============================================================================%
%
%	Right side
%
%============================================================================%
\begin{rightcolumn}
%---------------------------------------------------------------------------------------
%	PROFIL
%----------------------------------------------------------------------------------------
\cvsection{Profil}

\vspace{5pt}

\cvtext{Après un parcours professionnel riche en expériences diverses, j'ai décidé de vivre de ma passion qu'est l'informatique, et en particulier la programmation et le développement d'application.

Ayant renforcé et élargi mes connaissances acquises en autodidacte depuis une dizaine d'années avec \href{https://openclassrooms.com/fr/paths/68-developpeur-dapplication-python}{une formation \textit{'spécialité Python - Django'}}, je suis prêt à mettre toutes mes compétences à votre service pour participer au développement de projets dans ce domaine.}

\vspace{5pt}

%---------------------------------------------------------------------------------------
%	EXPERIENCE
%----------------------------------------------------------------------------------------
\cvsection{Parcours}

\cvnewevent
	{janvier 2019 - aujourd'hui}
	{Développeur d'application}
	{Freelance}
	{\vspace{-7.5pt} \cvlist{
 		\item Conception d'architecture adaptée aux impératifs.\\[-19pt]
 		\item Développement en respectant les règles de codage.\\[-19pt]
		\item Installation et rédaction de documentation.\\[-19pt]
 		\item Maintenance et évolution.
 	}}

\cvnewevent
	{septembre 2001 - août 2018}
	{Instituteur}
	{Éducation nationale}
	{\vspace{-7.5pt} \cvlist{
 		\item Enseignement en cycle 2.\\[-19pt]
 		\item Enseignement auprès de public non francophone.
 	}}

\cvnewevent
	{septembre 1996 - août 2001}
	{Technicien}
	{DGA- Direction générale de l'armement}
	{\vspace{-7.5pt} \cvlist{
 		\item Création et utilisation de dispositif électronique.\\[-19pt]
 		\item Maintenance d'appareil de mesure.
 	}}

\vspace{5pt}
%---------------------------------------------------------------------------------------
%	LICENSE
%----------------------------------------------------------------------------------------
\cvsection{Formation}

\cvneweventdip
	{OpenClassrooms - 2019}
	{\href{https://drive.google.com/open?id=129cnSQymnrcecfrcKD8EAMZQcxk_KHUE}{Développeur d'application - Python}}
	{\href{https://drive.google.com/open?id=129cnSQymnrcecfrcKD8EAMZQcxk_KHUE}{Diplôme niveau Bac+3/4.}}

\vspace{5pt}

\cvneweventdip
	{OpenClassrooms - 2017}
	{\href{https://drive.google.com/open?id=1mo3ZDXMeHR8A9eLl2mBg0eK4-HIIhfmk}{Domaine de la programmation et du web}}
	{\href{https://drive.google.com/open?id=1mo3ZDXMeHR8A9eLl2mBg0eK4-HIIhfmk}{Certifications de réussite.}}

\vspace{5pt}

\cvneweventdip
	{Éducation Nationale - 2003}
	{Institut de Formation des Maîtres}
	{Certificat de fin d'études.}
	
\vspace{5pt}

\cvneweventdip
	{DGA - 1996}
	{Centre de Formation des Armements Terrestres}
	{Brevet de Formation Technique et Bac STL.}

\vspace{15pt}
%---------------------------------------------------------------------------------------
%	WORKS
%----------------------------------------------------------------------------------------
\cvsection{Réalisations}

\cvrealase
	{2019}
	{\href{https://github.com/JBthePenguin/CanteraSongLyrics}{Application web pour l'association \textit{ARCAL 65}}}
	{\href{https://github.com/JBthePenguin/CanteraSongLyrics}{Un répertoire de chants et un calendrier}}
	{\href{https://github.com/JBthePenguin/CanteraSongLyrics}{Conception, développement, mise en ligne sur Heroku.}}
	{\cvlist {
		\item \textit{Backend}: Python - Django\\[-19pt]
		\item \textit{Frontend}: HTML/CSS - Bootstrap\\[-19pt]
		\item \textit{Base de données, Versionning}: PostgreSQL - Git\\[-19pt]
		\item \textit{Conception, Documentation, Installation}: Herokucli
	}}

\vspace{-17.5pt}

\cvrealase
	{2019}
	{\href{https://github.com/JBthePenguin/FarmManagement}{Application web pour \textit{GAEC Aux légumes de bouche}}}
	{\href{https://github.com/JBthePenguin/FarmManagement}{Un outil de gestion et d'analyse de l'exploitation}}
	{\href{https://github.com/JBthePenguin/FarmManagement}{Conception, développement avec test et mise en place sur le réseau local.}}
	{\cvlist {
		\item \textit{Backend}: Python - Django\\[-19pt]
		\item \textit{Frontend}: HTML/CSS - Bootstrap, Javascript - JQuery\\[-19pt]
		\item \textit{Base de données, Versionning, Tests}: SQLite - Git - TravisCI - selenium\\[-19pt]
		\item \textit{Installation}: VirtualBox - Ubuntu Server
	}}

\vspace{-17.5pt}

\cvrealase
	{2019}
	{\href{https://github.com/JBthePenguin/PoilAuNezWebApp}{Application web pour la compagnie \textit{Poil au nez}}}
	{\href{https://github.com/JBthePenguin/PoilAuNezWebApp}{Un site vitrine avec actualité, galerie et contact}}
	{\href{https://github.com/JBthePenguin/PoilAuNezWebApp}{Conception, développement, test et mise en ligne sur Heroku.}}
	{\cvlist {
		\item \textit{Backend}: Python - Django\\[-19pt]
		\item \textit{Frontend}: HTML/CSS - Bootstrap, Javascript - JQuery\\[-19pt]
		\item \textit{Base de données, Versionning, Tests}: PostgreSQL - Git - TravisCI - selenium\\[-19pt]
		\item \textit{Conception, Documentation, Installation}: UML - LaTeX - Herokucli
	}}

\vspace{-16pt}

\cvtext{
		\href{https://github.com/JBthePenguin}{\textit{\textcolor{darkcol}{\texttt{Et d'autres...}}}}
	}

\end{rightcolumn}
\end{paracol}
\end{document}
